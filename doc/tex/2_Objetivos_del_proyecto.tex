\capitulo{2}{Objetivos del proyecto}
En esta sección se describirán los distintos objetivos que tienen como meta el desarrollo del proyecto:
\section{Objetivos generales}
\begin{itemize}
  \item Diseñar las interfaces de usuario, la \textit{API}, la interacción entre visitante y guía, y el flujo de los datos.
  \item Desarrollar un asistente para visitar museos implementando \textit{localización indoor} y realidad aumentada.
  \item Desarrollar el asistente en un entorno web con la librería de \textit{javascript} \textit{ReactJS}.
  \item Desarrollar una \textit{API} que dé cohesión a todo el flujo de datos generado entre la generación de datos y la web.
  \item Desarrollar un sistema de posicionamiento \textit{indoor} fiable en base a librerías \textit{Open Source}
  \item Desarrollar una aplicación con las características mencionadas que sea  \textit{Open Source}
  
\end{itemize}
\section{Objetivos en investigación}
\begin{itemize}

  \item Obtener un sistema de precisión mejorado \textit{outdoor/indoor} con una precisión de 0,5 metros en horizontal.
\end{itemize}

\section{Objetivos técnicos}
\begin{itemize}
  \item Desarrollar una aplicación web en \textit{ReactJS} desde cero, que muestre en un mapa los usuarios con una actualización de la posición una vez por segundo (1 Hz).
  \item Crear salas virtuales desde el mapa para poder crear un sistema que indique cuántos usuarios hay en cada una.
  \item Dar a los usuarios experiencia de realidad aumentada implementando un \textit{canvas} con la obra de arte visualizada donde el guía pueda dibujar.
  \item Hacer un sistema de representación local para calcular la localización de los \textit{Tags} y representarlo en un sistema global en formato \textit{EPSG:3857}, que es el estándar de la exportación.
  
\end{itemize}
\section{Objetivos personales}
\begin{itemize}

  \item Obtener conocimientos sobre el posicionamiento en mapas a alta frecuencia.
  \item Obtener conocimientos avanzados sobre \textit{React}.
  \item Obtener conocimientos avanzados sobre la creación de \textit{endpoints}.
  \item Obtener conocimientos sobre \textit{geofencing}.
  \item Obtener conocimientos sobre gestión de \textit{canvas} con la librería \textit{P5}.
\end{itemize}

